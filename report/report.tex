\documentclass{article}
\usepackage{amsmath, amsfonts}
\usepackage{graphicx}
\usepackage{subcaption}
\usepackage{csvsimple}


\begin{document}

\title{Report Outline}
\author{Fabian}

\maketitle

\section{Introduction}
A short introduction to the field of load forecasting and its relevance.\\
Description of the competition.

\section{Review of related Work}
Review previous work from Gefcom 2012. Distinguish between temperature and load prediction.\\
Review papers with related approaches including but not limited to those that can be found under the following link: http://blog.drhongtao.com/2014/08/recommended-papers-for-gefcom2014-contestants.html

\section{Data}
Description of the data made available for the competition
\subsection{Temperature Data}

\begin{table}[h!]
\centering
\begin{tabular}{llllll}
Year & Mean        & Median & StD         & Min   & Max   \\
2001 & 60.72339045 & 62.2   & 15.16199974 & 16.96 & 93.44 \\
2002 & 61.61707306 & 63.64  & 16.25871076 & 21.6  & 95.76 \\
2003 & 59.86723744 & 61.92  & 15.72253502 & 13.96 & 91.56 \\
2004 & 60.60986794 & 63.6   & 16.20390846 & 14.6  & 91.6  \\
2005 & 60.72554795 & 62.62  & 16.47735821 & 15.12 & 97.4  \\
2006 & 61.17943836 & 62.22  & 14.84796583 & 20.48 & 94.96 \\
2007 & 61.76222831 & 63.76  & 16.05569468 & 17.96 & 97.68 \\
2008 & 60.73573315 & 61.84  & 15.58815285 & 17.92 & 96.12 \\
2009 & 60.33642466 & 62.28  & 15.88100762 & 12.64 & 93.12 \\
2010 & 60.16820091 & 63.16  & 18.50289712 & 16.6  & 97.44 \\
2011 & 62.6862927  & 65.32  & 16.28001634 & 17.84 & 96.48
\end{tabular}
\caption{Yearly basic statistics for average temperature over 25 weather stations in Fahrenheit}
\end{table}

The temperature data made available consists of 25 series of temperature data in Fahrenheit from 25 different weather statios dating from 01/01/2001 to 12/01/2011.\\
%$Corr_X(m) = \frac{\mathbb{E}\[X_t-\mu_x)(X_{t+m})\]}{\var_X}$
The Cross Correlation of the different temperature series with each other suggest that they can be explained to over 90\% by the first series. [include correlation plots]
If we assume a temperature of 60 degrees fahrenheit, that would allow for an error of maximum 6 degrees of fahrenheit or 3 degrees celsius. As we will see later this error is negligable given the inaccuracy of the temperature prediction.

\subsection{Load Data}
Some basic statistics computed by year.
\begin{table}[!h]
\centering
\begin{tabular}{llllll}
Year & Mean        & Median & StD         & Min  & Max   \\
2005 & 139.1157985 & 129.4  & 44.46885208 & 64.8 & 291.3 \\
2006 & 134.5321005 & 125    & 42.01147924 & 48.4 & 291.4 \\
2007 & 144.4574772 & 134.8  & 44.64622456 & 69   & 307.4 \\
2008 & 147.095526  & 134.9  & 45.84668098 & 72.5 & 295.9 \\
2009 & 149.1644292 & 139.2  & 44.95448884 & 64.4 & 303.8 \\
2010 & 161.1352055 & 150.3  & 52.87939039 & 72.4 & 315.6 \\
2011 & 148.4394041 & 135.25 & 50.07312759 & 16.1 & 317.5
\end{tabular}
\caption{Yearly basic statistics for Energy Load in Mega Watts}
\end{table}

\subsubsection{"Lag Analysis"}
The effect of the lag on the Time Series Correlation Coefficient can be demonstrated using the Autocorrelation function acf() built into R stats. Here five plots showing the Autocorrelation Function for different maximum lags are displayed: 
\begin{figure}[h!]
\centering
\includegraphics[width=.7\textwidth]{../data/analysis/acf-load-lag-var-days_font.pdf}
\caption{Plots of Autocorrelation Function estimates of hourly load data in Mega Watts for different maximum lags.}
\label{fig:load-acf}
\end{figure}

As can be seen the correlation diminishes exponentially up until a lag of 72h from a point in the time series, then stays more or less constant for up to 7-8 days whereafter it diminishes near linearly (up to a lag of 35 days). 
\subsection{Basic exploration with Time Series Analysis Methods}
Autocorrelation, include decompositions? (included in "Data")

\section{Feature 'Extraction'}
Description of the features obtained from the data. 
\subsection{Calendar Features}
hour, TOY vs. month

\section{Models}
Description of the Models (LM), GAM (more extensive), NN, RF

\section{Analysis}
What combination of features and models for temperature and load provide us with a good prediction accuracy with respect to Gefcom leaderboard?

\subsection{Error Measures}
Introduce error measures (RMSE, MAE, MAPE, PINBALL) and their differences here? Too late?

\subsection{Temperature Modeling}
\subsubsection{Data Processing}
average temperature vs. principal component

\subsubsection{Effect on Load Prediction}
Effect of temperature on load prediction evaluated using different methods:\\
Mean over past years (yearly lag), LM, GAM, NN, RF vs. true temperature

\subsection{Load Modeling}

\subsubsection{Performance of different Methods}
GAM, NN, RF 

\subsubsection{Model Formulas}
Evaluating the performance of different model formulas built with the features (GAM)


\section{Conclusion}
Draw conclusion based on the analysis done in the main part.

\end{document}